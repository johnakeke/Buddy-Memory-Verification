\documentclass[10pt,conference,compsoc]{IEEEtran}
\IEEEoverridecommandlockouts

\usepackage{cite}
\usepackage{amsmath,amssymb,amsfonts}
\usepackage{algorithmic}
\usepackage{graphicx}
\usepackage{textcomp}
\usepackage{xcolor}
\usepackage{latexsym}
\usepackage{isabelle,isabellesym}

\def\BibTeX{{\rm B\kern-.05em{\sc i\kern-.025em b}\kern-.08em
    T\kern-.1667em\lower.7ex\hbox{E}\kern-.125emX}}

\newtheorem{definition}{Definition}
\newtheorem{lemma}{Lemma}
\newtheorem{theorem}{Theorem}

\newcommand{\defprefix}{Definition}
\newcommand{\lemmaprefix}{Lemma}
\newcommand{\theoremprefix}{Theorem}

\newcommand{\equidom}[3]{{#1}\stackrel{#2}{\sim}{#3}}
\makeatletter
\newcommand{\superimpose}[2]{{\ooalign{$#1\@firstoftwo#2$\cr\hfil$#1\@secondoftwo#2$\hfil\cr}}}
\makeatother
\newcommand{\interf}{\leadsto}
\newcommand{\ninterf}{\mathrel{\mathpalette\superimpose{{\slash}{\leadsto}}}}

\renewcommand{\IEEEQED}{}

\isabellestyle{it}

\def\isabellecode{
	\tiny{}}
\let\endisabellecode=\endisabellecode

\def\isabellec{
	\isabellebody{}}
\let\endisabellec=\endisabellebody

\newcommand{\isacodeftsz}{\footnotesize}

\def\isabellebodyc{
	\minipage[c]{\textwidth}{}
	\isabellebody{}
}
\let\endisabellebodyc=\endminipage

\hyphenation{op-tical net-works semi-conduc-tor}

\begin{document}

\title{Formal Verification for A Buddy Allocation Model Specification}

\author{\IEEEauthorblockN{1\textsuperscript{st} Ke Jiang}
\IEEEauthorblockA{\textit{School of Computer Science and Engineering} \\
\textit{Nanyang Technological University}\\
Singapore, Singapore \\
johnjiang@ntu.edu.sg}
\and
\IEEEauthorblockN{2\textsuperscript{nd} Yongwang Zhao}
\IEEEauthorblockA{\textit{School of Computer Science and Engineering} \\
\textit{Beihang University}\\
\textit{Beijing Advanced Innovation Center for Big Data and Brain Computing} \\
\textit{Beihang University}\\
Beijing, China \\
zhaoyw@buaa.edu.cn}
\and
\IEEEauthorblockN{3\textsuperscript{rd} David San\'{a}n}
\IEEEauthorblockA{\textit{School of Computer Science and Engineering} \\
\textit{Nanyang Technological University}\\
Singapore, Singapore \\
sanan@ntu.edu.sg}
\and
\IEEEauthorblockN{4\textsuperscript{th} Yang Liu}
\IEEEauthorblockA{\textit{School of Computer Science and Engineering} \\
\textit{Nanyang Technological University}\\
Singapore, Singapore \\
yangliu@ntu.edu.sg}
}

\maketitle

\begin{abstract}
Buddy allocation algorithms are widely adopted by memory management systems to manage the address space accessed by applications. However, errors in any stage of the development process of the memory management component, from the specification to the implementation, may lead to critical issues in other components using it. Rigorous mathematical proofs provide strong assurance to the development process. We apply formal methods to ensure the correctness in the specification of a buddy allocation algorithm. In this paper, we use the interactive theorem prover Isabell/HOL to construct a specification for a buddy allocation algorithm consisting on operations to allocate and dispose memory areas. Thence we verify that the operations preserve key invariants over the memory to guarantee functional correctness of the algorithm. Finally, we verify that the operations also preserve integrity of the memory, therefore they do not affect other memory areas previously allocated.
\end{abstract}

\begin{IEEEkeywords}
Memory Specification, Formal Verification, Functional Correctness, Security.
\end{IEEEkeywords}


\section{Introduction}
Correctness of applications and libraries of a system using dynamic memory allocation rely on properties the memory have to preserve. Errors in any stage of the development process of the memory management component, may break those properties, leading to critical issues in the rest of the system. Along the last decades, formal methods have been successfully applied in the verification of critical systems. To improve confidence on the reliability of a memory management, verification of functional correctness and security properties is applied from the top specification layers down to the implementation and the machine code. 

Formal verification has been applied on memory managers going from very abstract models to more concrete ones. For instance, the work in~\cite{reg_higham} formalizes an abstraction of the memory management in term of write and read operations. They prove sequential consistency over the abstraction, so the memory is expressed as ordered sequence of reads and writes. Also focussing on a high level of abstraction, the work in~\cite{reg_blazy} provides a memory model for an imperative language, defining the necessary memory operations for the language at the specification and implementation levels. It defines an axiomatic reasoning framework, proving that the memory semantics satisfy such axiomatic rules. In a similar way~\cite{reg_mansky}, also provides an axiomatic proof system for a sequential memory model, bringing together a unified representation of the memory rules. In this sense, the work in~\cite{reg_mansky} proves that their memory rules also satisfy the ones in~\cite{reg_blazy}.


The work from~\cite{reg_peter} formally specify a buddy allocation algorithm verifying the absence of covert channels on it.


In this paper, we develop a specification for buddy allocation algorithms in the Isabelle/HOL proof assistant. The budy allocation algorithms provide to applications with two services for the allocation and disposal of memory blocks. Our specification must consists of algorithm details as specific as possible for the sake of capturing any feature in the algorithms. Then we specify and prove a number of properties for functional correctness of the algorithm. Finally, we show that the memory services preserve integrity of other memory areas previously allocated.

The following section briefly introduces the Isabelle/HOL verification environment. The next section is about the formalization of buddy allocation model including representation of our specification and proofs to properties for functional correctness. The verification of integrity for security is arranged in the following section. The last section is about the conclusions and future work.



\section{Background}
In this section, we give the necessary backgrounds \correct{for}{on the} buddy memory allocation algorithms and Isabelle/HOL theorem prover.

\subsection{Buddy Allocation Algorithms}
According to the algorithms, free blocks of all possible sizes are maintained in a multilevel list. As a result, it is easy to find a block in requested size if one is available. If there is no block in requested size, allocation operation will search for the first nonempty free list for blocks, whose size is bigger than requested size. Then a large block which is picked from the nonempty free list is split. It is divided into two smaller blocks and each smaller block becomes an unique buddy to the other. If the size of smaller block is still too large, one of the two smaller blocks is split again. The split process stops until one block in requested size appears. Then one of the available blocks is marked as occupied and returned to the requesting application. The others are added to the appropriate free lists.

When a block is deallocated, the algorithms check whether the block can be merged. A split block can only be merged with its unique buddy block, which then reforms the larger block they were split from. Among this process, the algorithms apply a flag bit strategy to quickly check if blocks belonging to the same parent are free, in order to decide whether to merge these blocks into one block. With this way, buddy memory system has small external fragmentation.

To implement buddy allocation algorithms, it applies two important data structures multilevel free linked-list and multilevel free bitmap. The block to be allocated or deallocated is directly picked from the head of linked-list or added into the tail. Bitmap uses 0 and 1 to implement the flag bit strategy. Fig. \ref{fig3} describes a moment in memory system using these two date structures.

\begin{figure}[htbp]
	\centering
	\includegraphics[width=0.5\textwidth]{fig3.pdf}
	\caption{Structures in Buddy Allocation Algorithms}
	\label{fig3}
\end{figure}

\subsection{Isabelle/HOL}
We use \correct{interactive theorem prover Isabelle/HOL}{the Isabelle/HOL interactive theorem prover}~\cite{reg_Isabelle/HOL} to conduct the specification and verification of the memory management. Isabelle/HOL is a higher order logic theorem prover, using a typed lambda calculus-like functional language for specifications.

Isabelle/HOL includes a specification for simple common types such as naturals (\emph{nat}), integers (\emph{int}) and booleans (\emph{bool}). It also specifies some composed data types like tuples, records, lists and sets that are parametrized with other types. Isabelle provides the interface \emph{datatype} for the creation of user defined types based on type constructors. 

Isabelle provides functions on predefined types to access their members or to provide additional operations over them. In the following we describe those functions and types that we use along this work. A tuple is denoted as (\emph{$t_1$} $\times$ \emph{$t_2$}), projection functions \emph{fst} and \emph{snd} respectively return elements $t_1$ and $t_2$. Lists are defined as a datatype with an empty construct denoted with \emph{NIL} or $[]$, and a concatenation construct denoted with $\#$, where $x\#xs$ adds $x$ to the front of $xs$. The $i$th component of a list $as$ is written as $as!i$. Isabelle/HOL provides functions for definite and indefinite descriptions. Definitive description is represented by $THE\ x.\ P\ x$ and returns the element uniquely described by the predicate $P$, else it returns and undefined value. Indefinite description is represented by $SOME\ x.\, P\ x$, selecting a random element from the predicate $P$ that must describe at least one element, else it returns an arbitrary value.

Isabelle/HOL allows \correct{user}{users} to create non-recursive specifications using the command \emph{definition}, and to create recursive specifications using commands \emph{primrec} and \emph{recursive}.

\section{Specification of Buddy Allocation Model}\label{sec:spec}
The specification of the buddy memory allocation consists of a model for the necessary data structures to represent the memory layouts, as well as the allocation and disposal operations. This specification follows the algorithms for the buddy memory management in Zephyr OS, which applies a quartering split over blocks.

\subsection{State Representation}
\label{statedes}
{In the specification, the state models the memory as a set of  quad-tree each of them representing a memory pool. At this level of the specification, we assume that applications work with our block entity, so requesting a memory block returns the block itself, which will be used later during the deallocation.}
{\footnotesize
\begin{align*}
&(set:\ 'a)\ tree\ =\ Leaf\ (L:\ 'a)\ | \\
&Node\ (LL:\ 'a\ tree)\ (LR:\ 'a\ tree)\ (RL:\ 'a\ tree)\ (RR:\ 'a\ tree)
\end{align*}
}
{We define the structure of a quad-tree inductively}.{A quad-tree is parametrized by a variable type $'a$ and it has two constructors}: \mycomment{this is not correct. a Leaf does not have a tree as a parameter, neither a Node has (it has four).}\emph{Leaf} and \emph{Node}. {A \emph{Leaf} is a terminal node storing values of the parametrized type $'a$, and a \emph{Node} has four (sub)trees that are built recursively.} Notations \emph{LL}, \emph{LR}, \emph{RL} and \emph{RR} return the corresponding subtrees of the \emph{Node} tree. Notation \textbf{set} {represents} a function that \mycomment{there is not a more natural way to write the following expression? how a function \textit{gathers [a/the] polymorphic notation?}, rewrite using normal senteces}gathers polymorphic notation \emph{'a} as a collection from all Leaf nodes.

In this specification, we use {the tuple} (\emph{block\_state\_type} $\times$ \emph{ID}) to instantiate the polymorphic {type} \emph{'a} in the quad-tree structure. Type \emph{block\_state\_type} {indicates} the usage state of a block {and it is constructed using an Isabelle/HOL \emph{datatype}}. It consists of two subtypes{:} \emph{ALLOC} and \emph{FREE} {The former is used to mark memory blocks that have been allocated to applications, whilst the latter is used to mark those unallocated blocks (hence they are free to be assigned to applications requesting memory)}. {The type \emph{ID} is a natural number representing the} address identifier of a memory block. Finally, type \emph{BlockTree} represents an instantiated quad-tree {in which terminal nodes represent allocated or free memory blocks identified with by a natural number. We will use indistinctively the terms of Leaf and memory block along the document}. Non terminal nodes \emph{Node} represent the splitting process of the algorithm. 
{\footnotesize
	\begin{align*}
	block\_state\_type\ &=\ FREE\ |\ ALLOC \\
	ID\ &=\ nat \\
	BlockTree\ &=\ (block\_state\_type\ \times\ ID)\ tree
	\end{align*}
}

{The allocation and free services are defined as a number of operations over the \emph{quad-tree} data structure representing the memory. These operations manipulate a \textsl{BlockTree}, accessing and modifying its structure and the data it stores}. Function \textbf{get\_level} takes{two \emph{BlockTree}, \emph{btree} and \emph{b}}, {and it returns a natural number \emph{level}} {that} represents the layer number {where} \emph{b} {is located} in \emph{btree} from the root node. {The level of a node with regards to itself is $0$, and if the \emph{blocktree} $b$ does not belongs to $btree$ the function also returns $0$.}  Functions \textbf{allocsets} and \textbf{freesets} take a \emph{BlockTree} \emph{btree} and returns a \emph{BlockTree} set \emph{aset} of all the Leaf nodes from \emph{btree} whose \emph{block\_state\_type} are well \emph{ALLOC} for the function \textbf{allocsets}, well \emph{FREE} for the function \textbf{freesets} \mycomment{the comma totally change the meaning of the sentence. With the comma you are saying that all leaf nodes from btree are ALLOC. To express what you wanted to say you need to remove the comma} \mycomment{Since both functions are similar it is better to put them together}. Function \textbf{freesets\_level} takes a \emph{Blocktree} \emph{btree} and a natural number $l$, and it returns a set with all the free Leaf nodes located at level $l$ in \emph{btree}. We use {the} notation \emph{idset} to represent the collection of all used \emph{IDs}. To create a new Leaf node, we {pick up as new leaf \emph{ID} any natural number not belonging to \emph{idset}}.

%Before introducing allocation model, we create a function \textbf{output\_level} \correct{that maps requested size to the allocation level in a quad-tree}{mapping levels of a }. The input parameters are a natural list \emph{blo\_list} and a natural \emph{rsize}. Static linked list \emph{blo\_list} is used to store all possible sizes of blocks in a quad-tree, and its indexes represent the levels of blocks located in this quad-tree. For example, the size of root node is 1024\emph{Mbytes} and the first level of quad-tree is 256\emph{Mbytes}, then \emph{blo\_list}!0 is equal to 1024 and \emph{blo\_list}!1 is 256. The \emph{blo\_list} is a strictly decreasing list to simulate the fact that the smaller the level is, the larger size the memory block owns. This function returns a natural index \emph{l} in \emph{blo\_list} with these constrains: the size it represents has to be greater than or equal to the size of requested block, and there is no smaller size that meets this condition. After that, the block size is picked up from \emph{blo\_list}, and then mapped to the correct level of the quad-tree by the index \emph{l} in \emph{blo\_list}. We use \emph{rlv} to represent the output. The definition of this mapping is as follows.

%\begin{definition} [Mapping Requested Size to Allocation Level]
%\label{mostsuitable}
%\end{definition}
%{\footnotesize
%\begin{align*}
%&output\_level\ blo\_list\ rsize \triangleq THE\ l.\ l < \vert blo\_list \vert \\
%&\wedge rsize \le blo\_list\ !\ l \\
%&\wedge ((\vert blo\_list \vert > 1 \wedge l < \vert blo\_list \vert - 1) \longrightarrow rsize > blo\_list\ !\ (l+1))
%\end{align*}
%}

\subsection{Allocation Model}

The allocation takes as input a set of quad-trees representing the available memory pools $bset$, and a natural number $s$ representing the requested size of the memory to allocate. If $s$ is bigger than the maximum size $\Omega$ for any memory pool then the allocation fails and the state is not modified. If $s$ is smaller or equal than $\Omega$, then it calls function \textbf{alloc} over $bset$ and the level $rlv$ containing blocks of size bigger or equal to $s$, given by $\Delta_s$ as defined in Section~\ref{sec:buddy}. \textbf{alloc} carries out the necessary operations to find a block of size bigger or equal than $s$, and conducts the necessary modifications on $bset$ as we describe below. 

We explain first functions that \textbf{alloc}  uses to obtain the level on a system with free blocks with capacity to allocate a requested size: \textbf{exists\_freelevel} and \textbf{freesets\_maxlevel}. We first provide a description of these functions. Function \textbf{exists\_freelevel} is a predicate taking as input a set of \emph{BlockTree} $bset$ (the collection of all quad-trees in memory system) and a natural number $rlv$, and returns true if there exist a \emph{BlockTree} $b\in bsets$ and a level $l$ in $b$ such that $l$ has at least a free node. Function \textbf{freesets\_maxlevel} has the same inputs, and it returns the maximum level less or equal than \emph{rlv} having at least a free node. Formally:

\begin{definition} [Existence of Free Leaf Nodes]
\end{definition}
\vspace{-7pt}
{\footnotesize
\begin{align*}
exists\_freelevel\ bset\ rlv &\triangleq \exists l.\ l \leq rlv \\
&\wedge \exists b \in bset.\ freesets\_level\ b\ l \ne \emptyset
\end{align*}
}
\vspace{-12pt}

\begin{definition} [Maximum Level of Free Leaf Nodes]
\end{definition}
\vspace{-7pt}	
{\footnotesize
\begin{align*}
&freesets\_maxlevel\ bset\ rlv \triangleq THE\ lmax.\ lmax \leq rlv \\
&\wedge \exists b \in bset.\ freesets\_level\ b\ lmax \neq \emptyset \\
&\wedge (\forall l \leq rlv.\ \exists b \in bset.\ freesets\_level\ b\ l \ne \emptyset \longrightarrow l \leq lmax)
\end{align*}
}
\vspace{-12pt}

During the allocation process, if there is not any free \emph{Leaf} node available at the best fit level $l$ for the requested size, but there exists a higher level $hl \geq l$ with free blocks, it is necessary to start the splitting process from $hl$ down to $l$. Function \textbf{split} recursively divides  $hl - l$ times a Leaf node \emph{b} into a Node tree \emph{btree}. It uses the function \textbf{divide} that takes a Leaf node \emph{b} and returns a new non-terminal Node tree \emph{n} with four terminal Leaf nodes. The division operation is always conducted on the leftmost subtree of $n$ marking it as allocated, while the rest are marked as \emph{FREE}. A simple example describing this process is shown in Fig.~\ref{fig:splitleaf}. We define the \emph{split} operation as follows.

\begin{figure*}[htbp]
	\centering
	\includegraphics[width=0.9\textwidth]{fig1.pdf}
	\caption{The progress of dividing a free leaf}
	\label{fig:splitleaf}
\end{figure*}

\begin{definition} [Split a Leaf Node]
\end{definition}
\vspace{-7pt}
{\footnotesize
\begin{align*}
split\ b\ lv \triangleq\ &if\ lv = 0\ then\ b \\
else\ Node\ &(split\ (LL\ divide\ b)\ (lv - 1))\ (LR\ divide\ b)\\ 
&(RL\ divide\ b)\ (RR\ divide\ b)
\end{align*}
}	
\vspace{-12pt}

In addition, the allocation use functions \textbf{set\_type}, \textbf{replace} and \textbf{replace\_leaf}. Function \emph{set\_type} takes a Leaf node \emph{b} and a target \emph{block\_state\_type} \emph{s} as inputs, and returns a leaf \emph{b'} resulting of changing the state of \emph{b} to \emph{s}. Function \emph{replace} takes a \emph{BlockTree} \emph{btree}, two Leaf nodes \emph{b} and \emph{b'} as inputs, and returns a tree replacing the Leaf node \emph{b} with \emph{b'} in \emph{btree}. Function \emph{replace\_leaf} takes a \emph{BlockTree} \emph{btree}, a Leaf node \emph{b} and a Node \emph{btr} as inputs, and returns the tree that replaces the Leaf node \emph{b} with \emph{btr} in \emph{btree}.


To allocate a block of size $s$ as described in Section~\ref{sec:buddy}, \textbf{alloc} firstly uses the function \emph{exists\_freelevel} to check whether there is a \emph{BlockTree} in \emph{bset} with free blocks in a a level $l \leq \emph{rlv}$. If there is not, then the allocation process fails and returns original \emph{bset}.  Otherwise, the function \emph{freesets\_maxlevel} returns the maximum level $lmax$ with $lmax \leq rlv$ containing free blocks. From here there are two options.

(a) there is a \emph{BlockTree} with free nodes at the requested level (hence\emph{lmax} is equal to thee requested level \emph{rlv}). In this case the allocation operation selects a \emph{BlockTree} \emph{btree} from $btree$ such that there are free blocks at level \emph{rlv}. After this, it randomly picks up a \emph{FREE} Leaf node \emph{l} in level \emph{rlv} from \emph{btree}, and obtains a new $btree'$ by modifying in $btree$ the type of \emph{l} as \emph{ALLOC} using the functions \emph{set\_type} and \emph{replace\_btree}. After that, the allocation returns an updated \emph{bset} by replacing the previous \emph{btree} with \emph{btree'}. Note that this execution branch do not adds any additional nodes into the tree structure, and only modifies the type of a terminal node at level \emph{rlv}.

(b) there is not a \emph{BlockTree} with free nodes at the requested level (hence \emph{lmax} is smaller to thee requested level \emph{rlv}). Since  \emph{lmax} is a lower level than \emph{rlv} means that free nodes have bigger size than what is requested. In that case it is necessary to modify the tree splitting a terminal Leaf $l$ at level \emph{lmax} down to level \emph{rlv} to create a block fitting better the requested size. The allocation operation selects a \emph{BlockTree} \emph{btree} from $btree$ such that there are free blocks at level \emph{lmax}. Then it selects randomly picks up a \emph{FREE} Leaf node \emph{l} in level \emph{lmax} from \emph{btree} and splits \emph{l} into a \emph{Node} \emph{btr} using the function \emph{split}. As explained above, \emph{split} \emph{``breaks"} the leaf \emph{l} into a new \emph{BlockTree} for which \emph{btr} is its root, and where there is block $b$ at depth $rlv - lmax$ from $btr$ that is allocated and its buddies are free. \textbf{alloc} then updates \emph{l} with \emph{btr} in  \emph{btree} using \emph{replace\_leaf} obtaining \emph{btree'}, to finally update \emph{bset} by replacing the previous \emph{btree} with \emph{btree'}. Finally, the operation returns updated Block set and \emph{True}.

Note that \textbf{alloc} modifies a free block into an allocated one, so the number of leafs remain unmodified in the branch (a); or breaks a free block into a \emph{Node} to eventually create a number of free blocks and a allocated block as shown in Fig.~\ref{fig:splitleaf}. In both cases the rest of leafs are from the original set do not change and there is a new allocated block that was not present. To obtain the block \textbf{alloc} allocates, it is only necessary to subtract the allocate nodes in the original set of free nodes from the allocate nodes in the resulting set.

\begin{definition} [Allocation Operation]
\end{definition}
\vspace{-7pt}
{\footnotesize
\begin{align*}
alloc\ &bset\ rlv \triangleq \\
&if\ exists\_freelevel\ bset\ rlv\ then \\
&\ \ \ \ lmax = freesets\_maxlevel\ bset\ rlv \\
&\ \ \ \ if\ lmax = rlv\ then \\
&\ \ \ \ \ \ \ \ btree = SOME\ b.\ b \in bset \wedge freesets\_level\ b\ rlv \ne \emptyset \\
&\ \ \ \ \ \ \ \ l = SOME\ l.\ l \in freesets\_level\ btree\ rlv \\
&\ \ \ \ \ \ \ \ btree' = replace\ btree\ l\ (set\_type\ l\ ALLOC) \\
&\ \ \ \ \ \ \ \ return\ (bset - \lbrace btree \rbrace \cup \lbrace btree' \rbrace, True) \\
&\ \ \ \ else \\
&\ \ \ \ \ \ \ \ btree = SOME\ b.\ b \in bset \wedge freesets\_level\ b\ lmax \ne \emptyset \\
&\ \ \ \ \ \ \ \ l = SOME\ l.\ l \in freesets\_level\ btree\ lmax \\
&\ \ \ \ \ \ \ \ btr' = split\ l\ (rlv - lmax) \\
&\ \ \ \ \ \ \ \ btree' = replace\_leaf\ btree\ l\ btr' \\
&\ \ \ \ \ \ \ \ return\ (bset - \lbrace btree \rbrace \cup \lbrace btree' \rbrace, True) \\
&else\ return\ (bset, False)
\end{align*}
}
\vspace{-17pt}

\subsection{Deallocation Model}

The deallocation process takes a block $b$ in a level $l$, and sets the state of $b$ to \emph{FREE}. During the deallocation process, if the state of all the buddies of $b$ is already $free$ and $l$ is not the root node, the buddy nodes are merged to avoid fragmentation. In the merging process, $b$ parent \emph{Node} is transformed into a \emph{Leaf} at level $l-1$ that is set as free, causing all the buddies to be removed from the quad-tree. Hence they are not at level $l$ anymore). The merging process is shown in Fig.~\ref{fig:merginfreeblocks}.

\begin{figure*}[htbp]
	\centering
	\includegraphics[width=0.9\textwidth]{fig2.pdf}
	\caption{The progress of merging all free memory blocks}
	\label{fig:merginfreeblocks}
\end{figure*}

Function \textbf{free} takes as input a set of \emph{BlockTree} $bset$ and a block to dispose \emph{b}. It firstly checks whether there is a $btree \in bset$ for which \emph{b} belongs to and that its state is \emph{ALLOC}. If the conditions are not met, the procedure fails returning the original \emph{bset}. If they are, the procedure picks up the  tree in \emph{bset} as \emph{btree} to which \emph{b} belongs to, which must be unique. After this, \emph{btree} is modified into \emph{btree'} where the type of \emph{b} is set to \emph{FREE} using the functions \emph{set\_type} and \emph{replace}. After that, the new tree is coalesced using the function \emph{merge} and \emph{bset} is updated replacing \emph{btree} with the new memory pool. The definition of deallocation operation is as follows.

\begin{definition} [Deallocation Operation]
\end{definition}
\vspace{-7pt}
{\footnotesize
\begin{align*}
free\ &bset\ b \triangleq \\
&if\ \exists btree \in bset.\ b \in set\ btree\ then \\
&\ \ \ \ if\ fst\ b = FREE\ then \\
&\ \ \ \ \ \ \ \ return\ (bset, False) \\
&\ \ \ \ else \\
&\ \ \ \ \ \ \ \ btree = THE\ t.\ t \in bset \wedge b \in set\ t \\
&\ \ \ \ \ \ \ \ btree' = replace\ btree\ b\ (set\_type\ b\ FREE) \\
&\ \ \ \ \ \ \ \ btree'' = merge\ btree' \\
&\ \ \ \ \ \ \ \ return\ (bset - \lbrace btree \rbrace \cup \lbrace btree'' \rbrace, True) \\
&else\ return\ (bset, False)
\end{align*}
}
\vspace{-12pt}

At this point, we have finished the specification of the allocation and disposal operations for a buddy memory allocator. The next section tackles the verification of functional and security properties over the model for the allocation and deallocation operations.
\section{Verification for Memory Model Specification}
To guarantee functional correctness of the formal model we proof a number of properties related to the transformations that the operations carry out over the memory structure. We try to answer the following questions: do the operations pick out the most suitable block from all the available blocks? is the state of the data structures representing the memory correctly set after executing the operations? are free blocks properly merged after a disposal operation? Answering these questions contributes to the construction of a reliable memory system.

Once the specification is finished, the preconditions $\&$ postconditions for functions as well as the invariants are to be raised up to ensure the functional correctness of the specification. We give preconditions and postconditions in the first place.

The following Def. \ref{pp1} gives such an implication: if there is not a quad-tree in \emph{blo\_set} that has free memory blocks whose level is less than or equal to \emph{rlv}, nothing is to be changed because of the allocation failure.

\begin{definition} [Allocation Failure] \\
	$\neg$ exists\_freelevel blo\_set rlv $\longrightarrow$ fst (alloc blo\_set rlv) = blo\_set
	\label{pp1}
\end{definition}

Then Def. \ref{pp2} and Def. \ref{pp3} respectively describe the block to be allocated no longer belongs to the free sets and is part of allocated sets during the direct allocation process.

\begin{definition} [Freesets for Direct Allocation] \\
	exists\_freelevel blo\_set rlv $\wedge$ freesets\_maxlevel blo\_set rlv = rlv $\longrightarrow$ \\
	\phantom{x} \hspace{10pt} $\exists$l. l $\in$ freesets blo\_set $\wedge$ l $\notin$ freesets fst (alloc blo\_set rlv) $\wedge$ \\
	\phantom{x} \hspace{10pt} freesets blo\_set = freesets fst (alloc blo\_set rlv) $\cup$ $\lbrace$l$\rbrace$
	\label{pp2}
\end{definition}

\begin{definition} [Allocsets for Direct Allocation] \\
	exists\_freelevel blo\_set rlv $\wedge$ freesets\_maxlevel blo\_set rlv = rlv $\longrightarrow$ \\
	\phantom{x} \hspace{10pt} $\exists$l. l $\notin$ allocsets blo\_set $\wedge$ l $\in$ allocsets fst (alloc blo\_set rlv) $\wedge$ \\
	\phantom{x} \hspace{10pt} allocsets fst (alloc blo\_set rlv) = allocsets blo\_set $\cup$ $\lbrace$l$\rbrace$
	\label{pp3}
\end{definition}

Next Def. \ref{pp4} says that a new block created from splitting a bigger block is allocated and belongs to the allocated sets during indirect allocation, which means \emph{split} operation is needed.

\begin{definition} [Allocsets for Indirect Allocation] \\
	exists\_freelevel blo\_set rlv $\wedge$ freesets\_maxlevel blo\_set rlv $\neq$ rlv $\longrightarrow$ \\
	\phantom{x} \hspace{10pt} $\exists$l. l $\notin$ allocsets blo\_set $\wedge$ l $\in$ allocsets fst (alloc blo\_set rlv) $\wedge$ \\
	\phantom{x} \hspace{10pt} allocsets fst (alloc blo\_set rlv) = allocsets blo\_set $\cup$ $\lbrace$l$\rbrace$
	\label{pp4}
\end{definition}

The next two Def. \ref{pp5} and Def. \ref{pp6} guarantee nothing to be changed during the deallocation process because of the non-existence of such a quad-tree that the block to be released belongs to or the type of the block to be freed is \emph{FREE}.

\begin{definition} [Deallocation failure 1] \\
	$\nexists$btree $\in$ blo\_set. b $\in$ tree.set btree $\longrightarrow$ fst (free blo\_set b) = blo\_set
	\label{pp5}
\end{definition}

\begin{definition} [Deallocation failure 2] \\
	$\exists$btree $\in$ blo\_set. b $\in$ tree.set btree $\wedge$ type b = FREE $\longrightarrow$ \\
	\phantom{x} \hspace{10pt} fst (free blo\_set b) = blo\_set
	\label{pp6}
\end{definition}

The last Def. \ref{pp7} tells that the block to be deallocated does not belong to allocated sets any more.

\begin{definition} [Allocsets for Deallocation Success] \\
	$\exists$btree $\in$ blo\_set. b $\in$ tree.set btree $\wedge$ type b $\neq$ FREE $\longrightarrow$ \\
	\phantom{x} \hspace{10pt} allocsets blo\_set = allocsets fst (free blo\_set b) $\cup$ $\lbrace$b$\rbrace$
	\label{pp7}
\end{definition}

After giving these definitions for preconditions and postconditions, we try to prove that our buddy allocation specification satisfies them so that it meets the functional expectations. The first theorem we prove as follows.

\begin{theorem}
The buddy allocation specification satisfies all the preconditions and postconditions above.
\end{theorem}

\begin{proof}
	All parts can be formally proved by induction on the type of a quad-tree, including leaf and node types. And the proofs on node type can be conducted by another induction which is on the height of the derivation.
\end{proof}

Through preconditions and postconditions, the specification can be proved that it partly follows the expectations of the algorithms. Next to guarantee that the algorithms pock out the most suitable block, two properties have to be proved: 1. the correctness of the mapping from the request memory block size to the level of the quad-tree. 2. the correctness of the quad-tree hierarchical structure. Here are these two properties.



 Function \textbf{L} is used to get the length of a list. As mentioned above, the smaller the level, the larger size the memory block.  Some lemmas ensure the correctness of this definition.

\begin{lemma}
	L blo\_list $>$ 0 $\wedge$ rsize $\leq$ blo\_list ! (L blo\_list - 1) $\longrightarrow$ output\_level blo\_list rsize = L blo\_list - 1
\end{lemma}

\begin{lemma}
	L blo\_list $>$ 1 $\wedge$ l $<$ L blo\_list - 1 $\wedge$ rsize $\leq$ blo\_list ! l $\wedge$ rsize $>$ blo\_list ! (l + 1) $\longrightarrow$ output\_level blo\_list rsize = l
\end{lemma}

\begin{proof}
	By unfolding the definition of output\_level, two aspects including existence and uniqueness have to be proved for the notion THE. The first part existence can be proved by the given assumptions. The remaining part uniqueness can be proved by the strictly decreasing blo\_list.
\end{proof}

Next, we introduce a lemma alone to prove the correctness of a quad-tree structure from the aspect of its levels. The \textbf{root} checks whether the tree is a root tree and \textbf{child} gives us all immediate child nodes.

\begin{lemma}
	(root btree $\longrightarrow$ get\_level btree = 0) $\wedge$ (get\_level btree = l $\wedge$ l $\geq$ 0 $\wedge$ chtree $\in$ child btree $\longrightarrow$ get\_level chtree = l + 1)
\end{lemma}

\begin{proof}
	This lemma can be formally proved by induction on the height of the derivation.
\end{proof}

Until now, we have already proved the correctness of the mapping operation and the hierarchical structure of a quad-tree. Then we can deduce the following theorem to guarantee the property of picking out the most suitable memory block.

\begin{theorem}
	The buddy allocation specification picks out the most suitable memory block and operates on the correct level in a quad-tree.
\end{theorem}

The algorithms are designed to avoid the fragmentation by merging operation which is invoked in the process of deallocation. Whether this operation is executed correctly can not be proved from the surface. In order to prove this property, we still start from the structural correctness of the quad-tree. Considering the fact that there is not such a node whose four child nodes are all leaves and their types are \emph{FREE} after merging operation, a definition to check whether a tree is like this can be constructed as follows. The \textbf{leaf} is to check whether the tree is a leaf.

\begin{definition} [Four Free Leaves Belong to The Same Node] \\
	is\_FFL btree $\equiv$ $\forall$chtree $\in$ child btree. leaf chtree $\wedge$ type chtree = FREE
\end{definition}

Fig. \ref{fig2} can explain this definition well. Just like the subtree in the lower left corner of the first picture, four child nodes are all leaves and their types are \emph{FREE}. Therefore, merging operation is necessary during the progress of deallocation to handle this situation. The following are the lemmas that ensure the non-existence of such \emph{FFL} trees after allocation and deallocation operations if non-existence of such trees in preconditions.

\begin{lemma}
	$\forall$b $\in$ blo\_set. $\neg$ is\_FFL b $\longrightarrow$ $\forall$b $\in$ fst (alloc blo\_set rlv). $\neg$ is\_FFL b
\end{lemma}

\begin{lemma}
	$\forall$b $\in$ blo\_set. $\neg$ is\_FFL b $\longrightarrow$ $\forall$b $\in$ fst (free blo\_set b). $\neg$ is\_FFL b
\end{lemma}

\begin{proof}
	Apply cases to these two lemmas after folding the definitions of allocation and deallocation operations. For each cases, they can be proved by induction on the height of the derivation.
\end{proof}

After memory initialization, assuming that all blocks to be allocated are the original ones which are not split, this beginning of the moment satisfies non-existence of \emph{FFL} trees among all quad-trees because all available blocks are seen as root trees. This assertion satisfies the assumptions in the implication expressions above. Therefore, through any execution orders of allocation and deallocation operations, the whole memory system satisfies non-existence of the \emph{FFL} trees. We have this theorem as follows.

\begin{theorem}
	The buddy allocation specification guarantees non-existence of FFL trees among all quad-trees.
\end{theorem}

In the end, we prove two significant properties: memory isolation and non-leakage. The first one is to prove non-existence of the overlap in the address spaces. Isolation in address spaces makes sure domains' memory blocks are not maliciously overwritten. Memory leakage means that available memory blocks (including occupied and free blocks) are getting less and less. Then non-leakage is to protect the integrity of address spaces.

Now we begin with the memory isolation. For the specification level, we provisionally use \emph{ID} to represent a contiguous addresses for a memory block. When we introduce real addresses into the specification, two things have to be proved: 1. the correctness of mapping function between a \emph{ID} and a true range of address; 2. the one-to-one uniqueness between them. In this paper, we are not going to introduce real addresses and we assume the above properties are all correct. Therefore, in this specification, isolation in address spaces means that all \emph{IDs} that leaves bring are different. What we prove to guarantee the difference of \emph{IDs} is the strategy of creating a new leaf that has been already introduced in the subsection 3.1.

The following definition tells that there are no two leaves with the same \emph{ID}. The \textbf{ID} gives the id the leaf brings. Firstly, we pick up any quad-tree \emph{b} from the tree collection \emph{blo\_set}. Then we select any leaf \emph{l} from this quad-tree. Our criterion is that there is not such a leaf \emph{l'} picked up from any quad-tree that is different from \emph{l} but has the same \emph{ID} with \emph{l}.

\begin{definition} [Different IDs] \\
	is\_different blo\_set $\equiv$ $\forall$b $\in$ blo\_set. $\forall$l $\in$ tree.set b. ($\nexists$l'. l' $\in$ tree.set (SOME b. b $\in$ blo\_set) $\wedge$ l' $\ne$ l $\wedge$ ID l' = ID l)
\end{definition}

Below are two lemmas that ensure this property holds during the procedures of allocation and deallocation if it holds in preconditions.

\begin{lemma}
	is\_different blo\_set $\longrightarrow$ is\_different fst (alloc blo\_set rlv)
\end{lemma}

\begin{lemma}
	is\_different blo\_set $\longrightarrow$ is\_different fst (free blo\_set b)
\end{lemma}

\begin{proof}
	Firstly apply cases to these two lemmas after folding the definitions of allocation and deallocation operations. Then for each cases, they can be proved by induction on the height of the derivation.
\end{proof}

With above lemmas of different \emph{IDs}, we can ensure all \emph{IDs} are different by using our strategy to create a new leaf. The following is the theorem that says this.

\begin{theorem}
	The buddy allocation specification ensures all IDs of leaves are different.
\end{theorem}

Finally, combined with the assumptions that mapping function between a \emph{ID} and a true range of address is correct, memory isolation in addresses can be proved.

Next is for the non-leakage of blocks. Result from we use the quad-tree structure and map all the memory blocks into the leaves of these trees, thence the non-leakage means that all the leaves (including occupied and free leaves) are in use. If we can infer that the quad-tree always maintains correct structure from the aspect of a relation between the number of nodes and leaves, then we can prove that all the leaves are in use and none leaf is forgotten. The first step is to prove a relation between the number of nodes and leaves in a quad-tree. Functions \textbf{Leaf} and \textbf{Node} give all the leaves and nodes in a quad-tree \emph{b}.

\begin{lemma}
	q-tree b: Num (Leaf b) = Num (Node b) $\times$ 3 + 1
\end{lemma}

\begin{proof}
	This lemma can be formally proved by induction on the type of quad-tree b, including leaf and node types. If quad-tree b is a leaf which means it is also a root, then the number of leaves is 1 and the number of nodes is 0. The lemma can be proved. If quad-tree b is a node with the inductive assumptions that this relation in the number of leaves and nodes holds for all the subtrees in b, this lemma still can be proved because of the quad-tree structure of b itself.
\end{proof}

Having established this relation, we use the following two lemmas to guarantee all the quad-trees during the procedures maintain this relation in the number of leaves and nodes.

\begin{lemma}
	$\forall$b $\in$ blo\_set. q-tree b $\longrightarrow$ $\forall$b $\in$ fst (alloc blo\_set rlv). q-tree b
\end{lemma}

\begin{lemma}
	$\forall$b $\in$ blo\_set. q-tree b $\longrightarrow$ $\forall$b $\in$ fst (free blo\_set b). q-tree b
\end{lemma}

\begin{proof}
	Fold the definitions of allocation and deallocation operations firstly. Then apply cases to these two lemmas. For each cases, they can be proved by induction on the height of the derivation.
\end{proof}

In the end, we can prove that all quad-trees holds this relation in the number of leaves and nodes. That is to say all leaves (including occupied and free leaves) are in use. Then considering the fact that all blocks are mapped into the leaves of these trees, we can ensure non-leakage of memory in our specification.

\begin{theorem}
	The buddy allocation specification guarantees any tree is a q-tree.
\end{theorem}

To sum up, in this section we introduce the quad-tree structure to simulate memory because of the buddy allocation algorithms. Then we give a specification for the algorithms including \textbf{alloc} and \textbf{free} operations. After that, we give proofs for functional correctness including preconditions and postconditions, the most suitable memory block, non-existence of the \emph{FFL} trees, memory isolation and non-leakage. Through these efforts, we give a functionally correct buddy memory model.
\section{Conclusions and Future Work}
In this paper, we have presented a specification for buddy allocation algorithms with two services for the allocation and disposal of memory blocks. Then we specify and prove the necessary properties for functional correctness of the specification regarding the blocks allocated and disposed and the structure of the memory layout. After that, we introduce integrity property for the security assurance of this specification. To achieve this goal, we design a state-machine security model with the concepts of interfering and state equivalence. Through instantiating this security model by adding interface functions to the buddy allocation specification, we finally prove that the instantiation security model satisfies integrity property.

As future work, we plan to extend the current model with real address spaces, multilevel free linked-list and multilevel free bitmap to this buddy allocation specification to provide a design-level specification. Furthermore, we are planning to extend this work to an implementation-level specification using a modelling language able to capture the semantic of system programming languages. In this process, refinement is necessary to guarantee property preservation between two adjacent level specifications.


\begin{thebibliography}{00}
	\bibitem{reg_sahebolamri}
	A. Sahebolamri, S. Constable and S. J. Chapin. A Formally Verified Heap Allocator, Electrical Engineering and Computer Science, 2018, p. 182.
	
	\bibitem{reg_knuth}
	D. E. Knuth. Dynamic storage allocation. The Art of Computer Programming, volume 1, section 2.5, 1968, p. 435–455.
	
	\bibitem{reg_mangano}
	F. Mangano, S. Duquennoy and N. Kosmatov. Formal verification of a memory allocation module of Contiki with FRAMA-C : A case study, the 11th International Conference on Risks and Security of Internet and Systems, 2016, p. 114–120.
	
	\bibitem{reg_noninterference}
	J. A. Goguen and J. Meseguer. Security Policies and Security Models, Proceedings of the IEEE Computer Society Symposium on Research in Security and Privacy, 1982, p. 11-20.
	
	\bibitem{reg_knowlton}
	K. C. Knowlton. A fast storage allocator, Commun. ACM, 1965, p. 623-624.
	
	\bibitem{reg_higham}
	L. Higham, J. Kawash and N. Verwaal. Defining and comparing memory consistency models, Proceedings of the 10th International Conference on Parallel and Distributed Computing Systems, 1997, p. 349–356.
	
	\bibitem{reg_mckusick}
	M. K. McKusick, K. Bostic, M. J. Karels, and J. S. Quarterman. The Design and Implementation of the 4.4 BSD Operating System, 1996.
	
	\bibitem{reg_marti}
	N. Marti, R. Affeldt and A. Yonezawa. Formal Verification of the Heap Manager of an Operating System Using Separation Logic, International Conference on Formal Engineering Methods, ICFEM: Formal Methods and Software Engineering, 2006, p. 400-419.
	
	\bibitem{reg_blazy}
	S. Blazy and X. Leroy. Formal Verification of a Memory Model for C-Like Imperative Languages, ICFEM: Formal Methods and Software Engineering, 2005, p. 280-299.
	
	\bibitem{reg_Isabelle/HOL}
	T. Nipkow, L. C. Paulson and M. Wenzel. Isabelle/HOL-A Proof Assistant for Higher-Order Logical, volume 2283 of LNCS. Springer-Verlag, 2002.
	
	\bibitem{reg_mansky}
	W. Mansky, G. Dmitri and S. Zdancewic. An Axiomatic Specification for Sequential Memory Models, Computer Aided Verification, July 2015, p. 413-428.
	
	\bibitem{reg_securitymodel}
	Y. Zhao, D. Sanan, F. Zhang and Y. Liu. Refinement-based Specification and Security Analysis of Separation Kernels, IEEE Transactions on Depandable and Secure Computing, Volume 16, Issue 1, January 2019, p. 127-141.
\end{thebibliography}

\end{document}
